\documentclass[]{article}
\usepackage{lmodern}
\usepackage{amssymb,amsmath}
\usepackage{ifxetex,ifluatex}
\usepackage{fixltx2e} % provides \textsubscript
\ifnum 0\ifxetex 1\fi\ifluatex 1\fi=0 % if pdftex
  \usepackage[T1]{fontenc}
  \usepackage[utf8]{inputenc}
\else % if luatex or xelatex
  \ifxetex
    \usepackage{mathspec}
  \else
    \usepackage{fontspec}
  \fi
  \defaultfontfeatures{Ligatures=TeX,Scale=MatchLowercase}
\fi
% use upquote if available, for straight quotes in verbatim environments
\IfFileExists{upquote.sty}{\usepackage{upquote}}{}
% use microtype if available
\IfFileExists{microtype.sty}{%
\usepackage{microtype}
\UseMicrotypeSet[protrusion]{basicmath} % disable protrusion for tt fonts
}{}
\usepackage[margin=1in]{geometry}
\usepackage{hyperref}
\hypersetup{unicode=true,
            pdftitle={Course Project: Regression models},
            pdfauthor={Koen Vermeulen},
            pdfborder={0 0 0},
            breaklinks=true}
\urlstyle{same}  % don't use monospace font for urls
\usepackage{color}
\usepackage{fancyvrb}
\newcommand{\VerbBar}{|}
\newcommand{\VERB}{\Verb[commandchars=\\\{\}]}
\DefineVerbatimEnvironment{Highlighting}{Verbatim}{commandchars=\\\{\}}
% Add ',fontsize=\small' for more characters per line
\usepackage{framed}
\definecolor{shadecolor}{RGB}{248,248,248}
\newenvironment{Shaded}{\begin{snugshade}}{\end{snugshade}}
\newcommand{\KeywordTok}[1]{\textcolor[rgb]{0.13,0.29,0.53}{\textbf{#1}}}
\newcommand{\DataTypeTok}[1]{\textcolor[rgb]{0.13,0.29,0.53}{#1}}
\newcommand{\DecValTok}[1]{\textcolor[rgb]{0.00,0.00,0.81}{#1}}
\newcommand{\BaseNTok}[1]{\textcolor[rgb]{0.00,0.00,0.81}{#1}}
\newcommand{\FloatTok}[1]{\textcolor[rgb]{0.00,0.00,0.81}{#1}}
\newcommand{\ConstantTok}[1]{\textcolor[rgb]{0.00,0.00,0.00}{#1}}
\newcommand{\CharTok}[1]{\textcolor[rgb]{0.31,0.60,0.02}{#1}}
\newcommand{\SpecialCharTok}[1]{\textcolor[rgb]{0.00,0.00,0.00}{#1}}
\newcommand{\StringTok}[1]{\textcolor[rgb]{0.31,0.60,0.02}{#1}}
\newcommand{\VerbatimStringTok}[1]{\textcolor[rgb]{0.31,0.60,0.02}{#1}}
\newcommand{\SpecialStringTok}[1]{\textcolor[rgb]{0.31,0.60,0.02}{#1}}
\newcommand{\ImportTok}[1]{#1}
\newcommand{\CommentTok}[1]{\textcolor[rgb]{0.56,0.35,0.01}{\textit{#1}}}
\newcommand{\DocumentationTok}[1]{\textcolor[rgb]{0.56,0.35,0.01}{\textbf{\textit{#1}}}}
\newcommand{\AnnotationTok}[1]{\textcolor[rgb]{0.56,0.35,0.01}{\textbf{\textit{#1}}}}
\newcommand{\CommentVarTok}[1]{\textcolor[rgb]{0.56,0.35,0.01}{\textbf{\textit{#1}}}}
\newcommand{\OtherTok}[1]{\textcolor[rgb]{0.56,0.35,0.01}{#1}}
\newcommand{\FunctionTok}[1]{\textcolor[rgb]{0.00,0.00,0.00}{#1}}
\newcommand{\VariableTok}[1]{\textcolor[rgb]{0.00,0.00,0.00}{#1}}
\newcommand{\ControlFlowTok}[1]{\textcolor[rgb]{0.13,0.29,0.53}{\textbf{#1}}}
\newcommand{\OperatorTok}[1]{\textcolor[rgb]{0.81,0.36,0.00}{\textbf{#1}}}
\newcommand{\BuiltInTok}[1]{#1}
\newcommand{\ExtensionTok}[1]{#1}
\newcommand{\PreprocessorTok}[1]{\textcolor[rgb]{0.56,0.35,0.01}{\textit{#1}}}
\newcommand{\AttributeTok}[1]{\textcolor[rgb]{0.77,0.63,0.00}{#1}}
\newcommand{\RegionMarkerTok}[1]{#1}
\newcommand{\InformationTok}[1]{\textcolor[rgb]{0.56,0.35,0.01}{\textbf{\textit{#1}}}}
\newcommand{\WarningTok}[1]{\textcolor[rgb]{0.56,0.35,0.01}{\textbf{\textit{#1}}}}
\newcommand{\AlertTok}[1]{\textcolor[rgb]{0.94,0.16,0.16}{#1}}
\newcommand{\ErrorTok}[1]{\textcolor[rgb]{0.64,0.00,0.00}{\textbf{#1}}}
\newcommand{\NormalTok}[1]{#1}
\usepackage{graphicx,grffile}
\makeatletter
\def\maxwidth{\ifdim\Gin@nat@width>\linewidth\linewidth\else\Gin@nat@width\fi}
\def\maxheight{\ifdim\Gin@nat@height>\textheight\textheight\else\Gin@nat@height\fi}
\makeatother
% Scale images if necessary, so that they will not overflow the page
% margins by default, and it is still possible to overwrite the defaults
% using explicit options in \includegraphics[width, height, ...]{}
\setkeys{Gin}{width=\maxwidth,height=\maxheight,keepaspectratio}
\IfFileExists{parskip.sty}{%
\usepackage{parskip}
}{% else
\setlength{\parindent}{0pt}
\setlength{\parskip}{6pt plus 2pt minus 1pt}
}
\setlength{\emergencystretch}{3em}  % prevent overfull lines
\providecommand{\tightlist}{%
  \setlength{\itemsep}{0pt}\setlength{\parskip}{0pt}}
\setcounter{secnumdepth}{0}
% Redefines (sub)paragraphs to behave more like sections
\ifx\paragraph\undefined\else
\let\oldparagraph\paragraph
\renewcommand{\paragraph}[1]{\oldparagraph{#1}\mbox{}}
\fi
\ifx\subparagraph\undefined\else
\let\oldsubparagraph\subparagraph
\renewcommand{\subparagraph}[1]{\oldsubparagraph{#1}\mbox{}}
\fi

%%% Use protect on footnotes to avoid problems with footnotes in titles
\let\rmarkdownfootnote\footnote%
\def\footnote{\protect\rmarkdownfootnote}

%%% Change title format to be more compact
\usepackage{titling}

% Create subtitle command for use in maketitle
\providecommand{\subtitle}[1]{
  \posttitle{
    \begin{center}\large#1\end{center}
    }
}

\setlength{\droptitle}{-2em}

  \title{Course Project: Regression models}
    \pretitle{\vspace{\droptitle}\centering\huge}
  \posttitle{\par}
    \author{Koen Vermeulen}
    \preauthor{\centering\large\emph}
  \postauthor{\par}
      \predate{\centering\large\emph}
  \postdate{\par}
    \date{5-9-2019}


\begin{document}
\maketitle

\subsection{Management summary}\label{management-summary}

In this project we will explore some features of cars that affect fuel
consumption in miles per gallon (MPG), transmission type in particular
(automatic vs.~manual).

We are looking a dataset of a collection of cars (mtcars - Motor Trend
Car Road Tests), and are interested in exploring the relationship
between a set of variables and Miles Per Gallon (MPG).

In particularly we want answer two questions: 1. Is an automatic or
manual transmission better for MPG? 2. What is the difference in MPG
between cars with an automatic and manual transmission?

\subsection{Loading \& preparating data}\label{loading-preparating-data}

Load the dataset and convert categorical variables to factors.

\begin{Shaded}
\begin{Highlighting}[]
\KeywordTok{library}\NormalTok{(ggplot2)}
\KeywordTok{data}\NormalTok{(mtcars) }\CommentTok{# loading the data}
\KeywordTok{head}\NormalTok{(mtcars, }\DataTypeTok{n=}\DecValTok{5}\NormalTok{) }\CommentTok{# first look at the data }
\KeywordTok{dim}\NormalTok{(mtcars) }\CommentTok{# dimensions are 32 records by 11 variables}
\NormalTok{class <-}\StringTok{ }\KeywordTok{as.data.frame}\NormalTok{(}\KeywordTok{sapply}\NormalTok{(mtcars, }\ControlFlowTok{function}\NormalTok{(x) }\KeywordTok{class}\NormalTok{(x)))}
\NormalTok{class}\OperatorTok{$}\NormalTok{unique <-}\StringTok{ }\KeywordTok{sapply}\NormalTok{(mtcars, }\ControlFlowTok{function}\NormalTok{(x) }\KeywordTok{length}\NormalTok{(}\KeywordTok{unique}\NormalTok{(x))); class}
\NormalTok{mtcars}\OperatorTok{$}\NormalTok{cyl <-}\StringTok{ }\KeywordTok{as.factor}\NormalTok{(mtcars}\OperatorTok{$}\NormalTok{cyl); mtcars}\OperatorTok{$}\NormalTok{vs <-}\StringTok{ }\KeywordTok{as.factor}\NormalTok{(mtcars}\OperatorTok{$}\NormalTok{vs)}
\NormalTok{mtcars}\OperatorTok{$}\NormalTok{am <-}\StringTok{ }\KeywordTok{factor}\NormalTok{(mtcars}\OperatorTok{$}\NormalTok{am); mtcars}\OperatorTok{$}\NormalTok{gear <-}\StringTok{ }\KeywordTok{factor}\NormalTok{(mtcars}\OperatorTok{$}\NormalTok{gear)}
\NormalTok{mtcars}\OperatorTok{$}\NormalTok{carb <-}\StringTok{ }\KeywordTok{factor}\NormalTok{(mtcars}\OperatorTok{$}\NormalTok{carb); class[class}\OperatorTok{$}\NormalTok{unique }\OperatorTok{<}\StringTok{ }\DecValTok{10}\NormalTok{,]}
\KeywordTok{levels}\NormalTok{(mtcars}\OperatorTok{$}\NormalTok{am) <-}\StringTok{ }\KeywordTok{c}\NormalTok{(}\StringTok{"Auto"}\NormalTok{, }\StringTok{"Manual"}\NormalTok{) }\CommentTok{# for easier analysis}
\end{Highlighting}
\end{Shaded}

\subsection{Exploratory data analysis}\label{exploratory-data-analysis}

To answer our questions we are initially interested in the relation
between the two parameters: transmission (am) and Miles per Gallon
(mpg). \textbf{Appendix 1} is a boxplot that compares automatic and
manual transmission MPG. The graph shows that the miles per gallon is
larger for manual than for automatic. But it still is possible that the
other variables play a role in the determination of MPG.

\textbf{Appendix 2} is a pairs graph that shows that MPG has
correlations with other variables than just \textbf{am}. To obtain a
more accurate model, we need to predict MPG with more variables than
\textbf{am} alone.

\subsection{Statistical inference}\label{statistical-inference}

\begin{Shaded}
\begin{Highlighting}[]
\NormalTok{ttest_MPG_AM <-}\StringTok{ }\KeywordTok{t.test}\NormalTok{(mtcars}\OperatorTok{$}\NormalTok{mpg }\OperatorTok{~}\StringTok{ }\NormalTok{mtcars}\OperatorTok{$}\NormalTok{am)}
\NormalTok{ttest_MPG_AM}\OperatorTok{$}\NormalTok{p.value}
\end{Highlighting}
\end{Shaded}

\begin{verbatim}
## [1] 0.001373638
\end{verbatim}

By performing the T-test we can check whether the difference in MPG
between the transmission types is 0. The results rejects the null
hypothesis that the difference between transmission types is 0.

\subsection{Regression analysis}\label{regression-analysis}

The \textbf{first model} (below) explains the MPG variability by the
transmission type (\textbf{am}) alone.

\begin{Shaded}
\begin{Highlighting}[]
\NormalTok{fit_MPG_AM <-}\StringTok{ }\KeywordTok{lm}\NormalTok{(mpg }\OperatorTok{~}\StringTok{ }\NormalTok{am, mtcars)}
\KeywordTok{summary}\NormalTok{(fit_MPG_AM)}\OperatorTok{$}\NormalTok{coef}
\KeywordTok{summary}\NormalTok{(fit_MPG_AM)}\OperatorTok{$}\NormalTok{r.squared}
\end{Highlighting}
\end{Shaded}

Eventhough the p-value is low (0.000285), the R-Squared is low as well
(0.3385). Therefore before making any conclusions on the effect of
transmission type on fuel efficiency, we have to take a look at models
inlcuding the other variables in the dataset.

The \textbf{second model} fits all variables on MPG.

\begin{Shaded}
\begin{Highlighting}[]
\NormalTok{fit_MPG_all <-}\StringTok{ }\KeywordTok{lm}\NormalTok{(mpg }\OperatorTok{~}\StringTok{ }\NormalTok{., mtcars)}
\KeywordTok{summary}\NormalTok{(fit_MPG_all)}
\end{Highlighting}
\end{Shaded}

Now, eventhough the R-squared has increased (0.8931), none of the
coefficients is significant at the 0.05 level. We have to find the
optimal model somewhere in the middle.

The \textbf{step-function} can be used to do the variable selection for
the optimal model.

\begin{Shaded}
\begin{Highlighting}[]
\NormalTok{fit_step <-}\StringTok{ }\KeywordTok{step}\NormalTok{(fit_MPG_all,}\DataTypeTok{direction=}\StringTok{"both"}\NormalTok{,}\DataTypeTok{trace=}\OtherTok{FALSE}\NormalTok{)}
\KeywordTok{summary}\NormalTok{(fit_step)}
\end{Highlighting}
\end{Shaded}

The new model has 4 variables (cylinders, horsepower, weight,
transmission). The R-squared value of 0.8659 confirms that this model
explains about 87\% of the variance in MPG. The p-values also are
statistically significantly because they have a p-value less than 0.05.
The coefficients conclude that increasing the number of cylinders from 4
to 6 with decrease the MPG by 3.03. Further increasing the cylinders to
8 with decrease the MPG by 2.16. Increasing the horsepower is decreases
MPG 3.21 for every 100 horsepower. Weight decreases the MPG by 2.5 for
each 1000 lbs increase. A Manual transmission improves the MPG by 1.81.

\subsubsection{Residual Analisys}\label{residual-analisys}

\textbf{Appendix 3} shows the residual plots of the last model
(fit\_best). These show: 1. The randomness of the Residuals vs.~Fitted
plot, which supports the assumption of independence 2. The points of the
Normal Q-Q plot following closely to the line, which concludes that the
distribution of residuals is normal 3. The Scale-Location plot random
distribution, which confirms the constant variance assumption 4. Since
all points are within the 0.05 lines, the Residuals vs.~Leverage, which
concludes that there are no outliers

\subsection{Conclusion}\label{conclusion}

There is a difference in MPG based on transmission type. A manual
transmission will have a slight MPG boost. However, it seems that
weight, horsepower, \& number of cylinders are more statistically
significant when determining MPG.

In the step-model the manual transmission improves the MPG by 1.81.

\subsection{Appendices}\label{appendices}

\subsubsection{Appendix 1}\label{appendix-1}

\begin{Shaded}
\begin{Highlighting}[]
  \KeywordTok{boxplot}\NormalTok{(mtcars}\OperatorTok{$}\NormalTok{mpg }\OperatorTok{~}\StringTok{ }\NormalTok{mtcars}\OperatorTok{$}\NormalTok{am, }
          \DataTypeTok{xlab=}\StringTok{"Transmission Type (0 = Automatic, 1 = Manual)"}\NormalTok{, }
          \DataTypeTok{ylab=}\StringTok{"MPG"}\NormalTok{,}
          \DataTypeTok{main=}\StringTok{"Boxplot: MPG by Transmission Type"}\NormalTok{)}
\end{Highlighting}
\end{Shaded}

\includegraphics{Version-1_files/figure-latex/unnamed-chunk-6-1.pdf}

\subsubsection{Appendix 2}\label{appendix-2}

\begin{Shaded}
\begin{Highlighting}[]
\KeywordTok{pairs}\NormalTok{(mtcars, }\DataTypeTok{panel =}\NormalTok{ panel.smooth, }\DataTypeTok{main =} \StringTok{"Pairs graph on mtcars"}\NormalTok{)}
\end{Highlighting}
\end{Shaded}

\includegraphics{Version-1_files/figure-latex/unnamed-chunk-7-1.pdf}

\subsubsection{Appendix 3}\label{appendix-3}

\begin{Shaded}
\begin{Highlighting}[]
\KeywordTok{par}\NormalTok{(}\DataTypeTok{mfrow =} \KeywordTok{c}\NormalTok{(}\DecValTok{2}\NormalTok{, }\DecValTok{2}\NormalTok{))}
\KeywordTok{plot}\NormalTok{(fit_step)}
\end{Highlighting}
\end{Shaded}

\includegraphics{Version-1_files/figure-latex/unnamed-chunk-8-1.pdf}


\end{document}
